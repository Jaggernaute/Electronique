
\newglossaryentry{noeud}{
    name={n{\oe}ud \'electrique},
    description={Point de connexion entre plusieurs conducteurs dans un circuit.}
}

\newglossaryentry{maille}{
    name={maille},
    description={Boucle ferm\'ee d'un circuit parcourue par le courant.}
}

\newglossaryentry{masse}{
    name={masse},
    description={R\'ef\'erence de potentiel d'un circuit, g\'en\'eralement \`a 0 volt.}
}

\newglossaryentry{courant_alternatif}{
    name={courant alternatif (AC)},
    description={Courant dont la valeur et la direction varient p\'eriodiquement dans le temps.}
}

\newglossaryentry{courant_continu}{
    name={courant continu (DC)},
    description={Courant dont la direction et l'intensit\'e restent constantes.}
}

\newglossaryentry{tension}{
    name={tension},
    description={Diff\'erence de potentiel \'electrique entre deux points d'un circuit.}
}

\newglossaryentry{intensite}{
    name={intensit\'e du courant},
    description={Quantit\'e de charge traversant une section de conducteur par unit\'e de temps.}
}

\newglossaryentry{charge}{
    name={charge \'electrique},
    description={Quantit\'e d'\'electricit\'e exprim\'ee en coulombs (C).}
}

\newglossaryentry{coulomb}{
    name={coulomb (C)},
    description={Unit\'e de charge \'electrique, \'equivalente \`a la charge transport\'ee par $6{,}24 \times 10^{18}$ \'electrons.}
}

\newglossaryentry{ampere}{
    name={amp\`ere (A)},
    description={Unit\'e d'intensit\'e du courant \'electrique : $1\,\text{A} = 1\,\text{C}/\text{s}$.}
}

\newglossaryentry{ampereheure}{
    name={amp\`ere-heure (Ah)},
    description={Quantit\'e de charge correspondant \`a un courant d'un amp\`ere pendant une heure.}
}

\newglossaryentry{valeur_efficace}{
    name={valeur efficace (RMS)},
    description={Racine carr\'ee de la moyenne du carr\'e d'une grandeur variable ; mesure la puissance \'equivalente en continu.}
}

\newglossaryentry{frequence}{
    name={fr\'equence (f)},
    description={Nombre de p\'eriodes par seconde d'un signal p\'eriodique, exprim\'e en hertz (Hz).}
}

\newglossaryentry{pulsation}{
    name={pulsation $\omega$},
    description={Vitesse angulaire d'un signal sinusoïdal, $\omega = 2\pi f$.}
}

\newglossaryentry{phase}{
    name={phase $\phi$},
    description={D\'ecalage angulaire entre deux grandeurs sinusoïdales.}
}

\newglossaryentry{facteur_puissance}{
    name={facteur de puissance $\cos\phi$},
    description={Rapport entre la puissance active et la puissance apparente d'un circuit en r\'egime AC.}
}

% --- Grandeurs \'electriques ---
\newglossaryentry{resistance}{
    name={r\'esistance (R)},
    description={Opposition au passage du courant \'electrique ; mesur\'ee en ohms $\Omega$.}
}

\newglossaryentry{resistivite}{
    name={r\'esistivit\'e $\rho$},
    description={Propri\'et\'e d'un mat\'eriau \`a s'opposer au passage du courant \'electrique.}
}

\newglossaryentry{conductivite}{
    name={conductivit\'e $\sigma$},
    description={Grandeur inverse de la r\'esistivit\'e ; mesure la facilit\'e du passage du courant.}
}

\newglossaryentry{conductance}{
    name={conductance (G)},
    description={Grandeur inverse de la r\'esistance, exprim\'ee en siemens (S).}
}

\newglossaryentry{impedance}{
    name={imp\'edance (Z)},
    description={Grandeur complexe reliant tension et courant en r\'egime sinusoïdal : $Z = R + jX$.}
}

\newglossaryentry{admittance}{
    name={admittance (Y)},
    description={Inverse de l'imp\'edance, $Y = 1/Z = G + jB$.}
}

\newglossaryentry{reactance}{
    name={r\'eactance (X)},
    description={Partie imaginaire de l'imp\'edance, caract\'erisant l'opposition au courant alternatif.}
}

\newglossaryentry{susceptance}{
    name={susceptance (B)},
    description={Partie imaginaire de l'admittance, inverse de la r\'eactance.}
}

\newglossaryentry{inductance}{
    name={inductance (L)},
    description={Propri\'et\'e d'un circuit \`a s'opposer aux variations de courant, mesur\'ee en henrys (H).}
}

\newglossaryentry{capacite}{
    name={capacit\'e (C)},
    description={Aptitude d'un composant \`a stocker de l'\'energie \'electrique sous forme de champ \'electrique.}
}

\newglossaryentry{effet_joule}{
    name={effet Joule},
    description={Transformation de l'\'energie \'electrique en chaleur dans un conducteur travers\'e par un courant.}
}

% --- Composants ---
\newglossaryentry{potentiometre}{
    name={potentiom\`etre},
    description={R\'esistance variable permettant de r\'egler une tension ou un courant.}
}

\newglossaryentry{rheostat}{
    name={rh\'eostat},
    description={R\'esistance variable utilis\'ee pour ajuster le courant dans un circuit.}
}

\newglossaryentry{esr}{
    name={r\'esistance \'equivalente s\'erie (ESR)},
    description={R\'esistance parasite interne d'un condensateur, mod\'elisant ses pertes.}
}

\newglossaryentry{condensateur}{
    name={condensateur},
    description={Composant stockant de l'\'energie sous forme de charge \'electrique.}
}

\newglossaryentry{supercondensateur}{
    name={supercondensateur (ultracondensateur)},
    description={Condensateur de tr\`es forte capacit\'e utilis\'e pour le stockage d'\'energie.}
}

\newglossaryentry{diode}{
    name={diode},
    description={Composant ne laissant passer le courant que dans un seul sens.}
}

\newglossaryentry{zener}{
    name={diode Zener},
    description={Diode utilis\'ee pour la r\'egulation de tension grâce \`a son effet de claquage contrôl\'e.}
}

\newglossaryentry{led}{
    name={LED (diode \'electroluminescente)},
    description={Diode \'emettant de la lumi\`ere lorsqu'elle est polaris\'ee dans le sens direct.}
}

\newglossaryentry{bjt}{
    name={BJT (transistor bipolaire)},
    description={Transistor command\'e par le courant de base ; Bipolar Junction Transistor.}
}

\newglossaryentry{mosfet}{
    name={MOSFET (transistor \`a effet de champ)},
    description={Transistor command\'e par la tension de grille ; Metal-Oxide–Semiconductor FET.}
}

\newglossaryentry{thyristor}{
    name={thyristor},
    description={Composant semi-conducteur de puissance contrôl\'e par une impulsion de gâchette.}
}

% --- Lois et th\'eor\`emes ---
\newglossaryentry{loi_ohm}{
    name={loi d'Ohm},
    description={Relation fondamentale : $U = R \times I$.}
}

\newglossaryentry{loi_noeuds}{
    name={loi des nœuds (KCL)},
    description={La somme des courants entrants dans un nœud est \'egale \`a la somme des courants sortants.}
}

\newglossaryentry{loi_mailles}{
    name={loi des mailles (KVL)},
    description={La somme des tensions dans une maille ferm\'ee est nulle.}
}

\newglossaryentry{thevenin}{
    name={th\'eor\`eme de Th\'evenin},
    description={Tout r\'eseau lin\'eaire peut être remplac\'e par une source de tension \'equivalente et une r\'esistance.}
}

\newglossaryentry{norton}{
    name={th\'eor\`eme de Norton},
    description={Tout r\'eseau lin\'eaire peut être remplac\'e par une source de courant \'equivalente et une r\'esistance.}
}

\newglossaryentry{millman}{
    name={th\'eor\`eme de Millman},
    description={Permet de calculer la tension commune \`a plusieurs branches parall\`eles aliment\'ees par diff\'erentes sources.}
}

% --- Circuits AC ---
\newglossaryentry{phasor}{
    name={phasor},
    description={Repr\'esentation complexe d'une grandeur sinusoïdale en amplitude et phase.}
}

\newglossaryentry{fresnel}{
    name={diagramme de Fresnel},
    description={Repr\'esentation graphique des grandeurs sinusoïdales sous forme vectorielle.}
}

\newglossaryentry{resonance}{
    name={r\'esonance},
    description={Ph\'enom\`ene où l'imp\'edance d'un circuit RLC est minimale et le courant maximal.}
}

\newglossaryentry{facteur_qualite}{
    name={facteur de qualit\'e $Q$},
    description={Mesure de la s\'electivit\'e d'un circuit r\'esonant.}
}

\newglossaryentry{filtre_passe_bas}{
    name={filtre passe-bas},
    description={Circuit laissant passer les basses fr\'equences et att\'enuant les hautes.}
}

\newglossaryentry{filtre_passe_haut}{
    name={filtre passe-haut},
    description={Circuit laissant passer les hautes fr\'equences et att\'enuant les basses.}
}

\newglossaryentry{filtre_passe_bande}{
    name={filtre passe-bande},
    description={Circuit ne laissant passer qu'une bande de fr\'equences autour d'une fr\'equence centrale.}
}

\newglossaryentry{fonction_transfert}{
    name={fonction de transfert $H(j\omega)$},
    description={Rapport complexe entre la sortie et l'entr\'ee d'un syst\`eme en r\'egime sinusoïdal.}
}

% --- Transformateurs et \'electromagn\'etisme ---
\newglossaryentry{induction}{
    name={induction \'electromagn\'etique},
    description={Production d'une tension dans un circuit par variation du flux magn\'etique.}
}

\newglossaryentry{flux_magnetique}{
    name={flux magn\'etique $(\Phi)$},
    description={Quantit\'e de champ magn\'etique traversant une surface donn\'ee.}
}

\newglossaryentry{rapport_transformation}{
    name={rapport de transformation},
    description={Rapport entre le nombre de spires et les tensions primaire/secondaire d'un transformateur.}
}

\newglossaryentry{rendement}{
    name={rendement $\eta$},
    description={Rapport entre la puissance utile et la puissance absorb\'ee d'un dispositif.}
}

\newglossaryentry{courants_foucault}{
    name={courants de Foucault},
    description={Courants induits dans les conducteurs soumis \`a un champ magn\'etique variable.}
}

% --- \'electronique de puissance ---
\newglossaryentry{redressement}{
    name={redressement},
    description={Conversion du courant alternatif en courant continu.}
}

\newglossaryentry{filtrage}{
    name={filtrage},
    description={Att\'enuation des ondulations apr\`es redressement, souvent par condensateur.}
}

\newglossaryentry{regulation}{
    name={r\'egulation de tension},
    description={Maintien d'une tension stable malgr\'e les variations de charge ou d'entr\'ee.}
}

\newglossaryentry{pwm}{
    name={PWM (modulation de largeur d'impulsion)},
    description={Technique de commande consistant \`a moduler la dur\'ee d'impulsion d'un signal pour contrôler la puissance moyenne.}
}

% --- RF et hyperfr\'equences ---
\newglossaryentry{rf}{
    name={RF (radiofr\'equence)},
    description={Domaine des fr\'equences sup\'erieures \`a quelques kHz jusqu'aux GHz, utilis\'e pour la communication.}
}

\newglossaryentry{hf}{
    name={HF (haute fr\'equence)},
    description={Bande de fr\'equences comprises entre 3 et 30 MHz.}
}

\newglossaryentry{microondes}{
    name={micro-ondes},
    description={Ondes \'electromagn\'etiques de fr\'equence entre 300 MHz et 300 GHz.}
}

\newglossaryentry{ligne_transmission}{
    name={ligne de transmission},
    description={Structure guidant une onde \'electromagn\'etique entre deux points.}
}

\newglossaryentry{guide_ondes}{
    name={guide d'ondes},
    description={Conduit m\'etallique destin\'e \`a guider les ondes \'electromagn\'etiques.}
}

\newglossaryentry{slti}{
    name={Syst\`eme Lin\'eaire et Invariant dans le Temps (SLIT)},
    description={Syst\`eme dont la r\'eponse est lin\'eaire et ind\'ependante du temps.}
}

\newglossaryentry{syntonisation}{
    name={syntonisation},
    description={Ajustement de deux circuits \`a la même fr\'equence de r\'esonance.}
}

\newacronym{ac_accro}{AC}{Courant alternatif}
\newacronym{dc_accro}{DC}{Courant continu}
\newacronym{rms_accro}{RMS}{Root Mean Square}
\newacronym{esr_accro}{ESR}{Equivalent Series Resistance}
\newacronym{bjt_accro}{BJT}{Bipolar Junction Transistor (transistor bipolaire)}
\newacronym{mosfet_accro}{MOSFET}{Metal-Oxide–Semiconductor Field-Effect Transistor}
\newacronym{pwm_accro}{PWM}{Pulse Width Modulation (modulation de largeur d'impulsion)}
\newacronym{rf_accro}{RF}{Radiofr\'equence}
\newacronym{hf_accro}{HF}{Haute fr\'equence}
\newacronym{slti_accro}{SLTI}{Syst\`eme Lin\'eaire et Invariant dans le Temps}
