%% Type de mémoire : licence, master, doctorat
\NewExpandableDocumentCommand{\ThesisType}{}{L3 EEEA}

%% Titre du mémoire (exactement comme dans le sujet officiel)
\NewExpandableDocumentCommand{\ThesisTitle}{}{Électronique et Télécommunications}

%% Version en texte brut pour les métadonnées PDF
\NewExpandableDocumentCommand{\ThesisTitlePlaintext}{}{Electronique et Telecommunications}

%% Titre pour la page de garde (formatage personnalisé)
\NewExpandableDocumentCommand{\ThesisTitleFront}{}{
    \ThesisTitle\\
    {\Huge\color{gray}vPre 0.0.1}
}

%% Auteur du mémoire
\NewExpandableDocumentCommand{\ThesisAuthor}{}{\textcolor{red}{Jaggi}}
%% Version en texte brut pour les métadonnées PDF
\NewExpandableDocumentCommand{\ThesisAuthorPlaintext}{}{Alex Videcoq}

%% Année de soumission
\NewDocumentCommand{\YearSubmitted}{}{2025}
%% Année de dernière révision (décommenter si différente de \YearSubmitted)
% \NewDocumentCommand{\YearRevision}{}{2025}

%% Université
\NewDocumentCommand{\University}{}{Université de Rennes}

%% Nom du département ou institut où le travail a été confié
\NewDocumentCommand{\Department}{}{\textcolor{red}{ISTIC}}

%% Type d’unité : Département ou Institut
\NewDocumentCommand{\DeptType}{}{UFR}

%% Directeur de mémoire : nom, prénom et titres
\NewDocumentCommand{\Supervisor}{}{\textcolor{red}{Prof. Proton}}
%% Co-directeur de mémoire : nom, prénom et titres (décommenter si applicable)
% \NewDocumentCommand{\CoSupervisor}{}{Prof. Matthieu Davy}

%% Département/Institut du directeur
\NewDocumentCommand{\SupervisorsDepartment}{}{\textcolor{red}{ISTIC}}

%% Filière et spécialisation
\NewDocumentCommand{\StudyProgramme}{}{\textcolor{red}{\'Electronique et Télécommunications}}

%% Résumé (80-200 mots recommandés)
\NewDocumentCommand{\Abstract}{}{
    Principes fondamentaux et avancés de l’électronique et des télécommunications, avec un accent particulier sur la théorie des circuits RF/micro-ondes, la conception de systèmes numériques, les composants hyperfréquences, et les techniques de mesure et de sécurité. Il couvre les lignes de transmission, les guides d’ondes, les amplificateurs opérationnels, les systèmes numériques, ainsi que les applications des micro-ondes dans les communications et la transmission d’énergie sans fil.
}

%% Sujet (courte description pour les métadonnées PDF)
\NewExpandableDocumentCommand{\Subject}{}{Électronique et Télécommunications, RF/Micro-ondes, Systèmes numériques}

%% Mots-clés (3-7)
\NewExpandableDocumentCommand{\Keywords}{}{%
    \textcolor{red}{RF, micro-ondes, amplificateurs, lignes de transmission, systèmes numériques, sécurité, mesures}
}

%% Version en texte brut pour les métadonnées PDF
\NewExpandableDocumentCommand{\KeywordsPlaintext}{}{%
    RF, micro-ondes, amplificateurs, lignes de transmission, systèmes numériques, sécurité, mesures
}
